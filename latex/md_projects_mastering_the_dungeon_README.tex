This project contains the code we used in our paper\+:

\href{https://arxiv.org/abs/1711.07950}{\tt Mastering the Dungeon\+: Grounded Language Learning by Mechanical Turker Descent}

Zhilin Yang, Saizheng Zhang, Jack Urbanek, Will Feng, Alexander H. Miller, Arthur Szlam, Douwe Kiela, Jason Weston

\subsection*{Requirements}

Python 3.\+6, Py\+Torch 0.\+2, spacy

To install {\ttfamily spacy} and download related packages\+: 
\begin{DoxyCode}
python -m pip install spacy
python -m spacy.en.download all
\end{DoxyCode}


\subsection*{Get the Data}

Go to the Parl\+AI root directory. Create a data directory if it does not exist. 
\begin{DoxyCode}
mkdir data
\end{DoxyCode}


Then we can download the data 
\begin{DoxyCode}
cd data
wget http://parl.ai/downloads/mastering\_the\_dungeon/mastering\_the\_dungeon.tgz
tar -xvzf mastering\_the\_dungeon.tgz
\end{DoxyCode}


\subsubsection*{Data Organization}

The dataset is organized as follows. {\ttfamily data/graph\+\_\+world2} contains the pilot study data, where each file ending with {\ttfamily .pkl} is a pickle file of an example. {\ttfamily data/graph\+\_\+world2\+\_\+v$\ast$\+\_\+r$\ast$} contains a pickle file {\ttfamily filtered\+\_\+data\+\_\+list.\+pkl} storing the data collected in a specific round of a specific setting. The number after {\ttfamily r} indicates the round index. The number after {\ttfamily v} indicates the setting, where {\ttfamily v13} means {\ttfamily M\+TD limit}, {\ttfamily v14} means {\ttfamily M\+TD}, {\ttfamily v15} means {\ttfamily M\+TD limit w/o model}, and {\ttfamily B\+A\+S\+E\+L\+I\+N\+E\+\_\+2} means the baseline. For example, {\ttfamily data/graph\+\_\+world2\+\_\+v15\+\_\+r2/filtered\+\_\+data\+\_\+list.\+pkl} contains the data collected in the second round of the setting {\ttfamily M\+TD limit w/o model}.

\subsubsection*{Splitting the Data}

We split the dataset before doing training and evaluation. Assuming you are under the Parl\+AI root directory, 
\begin{DoxyCode}
cd projects/mastering\_the\_dungeon/mturk/tasks/MTD
python run.py --split
\end{DoxyCode}


This will split the data from different rounds under different settings into training and test sets, while ensuring the number of training examples to be the same.

\subsection*{Training and Evaluation}

There are three steps\+: 1) start a G\+PU placeholder; 2) run the training and evaluation jobs; 3) terminate the G\+PU placeholder. We will illustrate the usage first on one single G\+PU, and later on a cluster with Slurm installed.

\subsubsection*{Start G\+PU Placeholder}

Open a new window (e.\+g. a screen session), and go to the Parl\+AI root directory. 
\begin{DoxyCode}
cd projects/mastering\_the\_dungeon/projects/graph\_world2
python train.py --job\_num 0
\end{DoxyCode}


This will start a G\+PU placeholder that accepts new training jobs on the fly.

\subsubsection*{Run Training Jobs}

Open another window (e.\+g. another screen session), and go to the Parl\+AI root directory.

\#\#\#\# Training 
\begin{DoxyCode}
cd projects/mastering\_the\_dungeon/mturk/tasks/MTD
python run.py --train [--seq2seq]
\end{DoxyCode}


This will train the models using A\+C-\/\+Seq2\+Seq (by default) or Seq2\+Seq (when \hyperlink{namespaceseq2seq}{seq2seq} is specified).

\#\#\#\# Evaluation 
\begin{DoxyCode}
cd projects/mastering\_the\_dungeon/mturk/tasks/MTD
python run.py --eval [--seq2seq] [--constrain]
\end{DoxyCode}


This will evaluate the models just trained. The option {\ttfamily constrain} indicates using constrained decoding or not; turning it on is recommended.

\#\#\#\# Breakdown by Round 
\begin{DoxyCode}
cd projects/mastering\_the\_dungeon/mturk/tasks/MTD
python run.py --rounds\_breakdown [--seq2seq]
\end{DoxyCode}


In this experiment, we will see the performance of agents in different rounds.

\#\#\#\# Breakdown by Dataset 
\begin{DoxyCode}
cd projects/mastering\_the\_dungeon/mturk/tasks/MTD
python run.py --data\_breakdown [--seq2seq]
\end{DoxyCode}


In this experiment, we will see the performance of agents trained on one dataset and evaluated on another dataset.

\#\#\#\# Ablation Study 
\begin{DoxyCode}
cd projects/mastering\_the\_dungeon/mturk/tasks/MTD
python run.py --ablation
\end{DoxyCode}


This will do the ablation study by considering removing the counter feature and the room embeddings.

\subsubsection*{Terminate G\+PU Placeholder}

After all training and evaluation jobs are finished, we can now terminate the placeholders. Go back to the placeholder window, and press {\ttfamily Ctrl+C} to terminate the process.

\subsubsection*{Running on G\+PU Cluster}

There are a few changes to be made when running on a G\+PU cluster. In this section we assume \href{https://slurm.schedmd.com/}{\tt Slurm} is installed on the cluster.

When creating the G\+PU placeholder, we use the following commands\+: 
\begin{DoxyCode}
cd projects/mastering\_the\_dungeon/projects/graph\_world2
python gen\_sbatch\_script.py --num\_gpus <num\_gpus> --slurm
./batch\_holder.sh
\end{DoxyCode}
 where {\ttfamily num\+\_\+gpus} is the number of G\+P\+Us to use.

For training and evaluation, we need to add an additional option {\ttfamily -\/-\/num\+\_\+machines} to every command. For example, training now becomes\+: 
\begin{DoxyCode}
cd projects/mastering\_the\_dungeon/mturk/tasks/MTD
python run.py --train --num\_machines <num\_gpus> [--seq2seq]
\end{DoxyCode}


To terminate the G\+PU placeholder, simply cancel all jobs (this assumes you are not running other jobs using Slurm) 
\begin{DoxyCode}
scancel -u <my\_username\_on\_slurm>
\end{DoxyCode}
 