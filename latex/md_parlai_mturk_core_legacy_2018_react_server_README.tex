{\ttfamily parlai/mturk/core/react\+\_\+server} contains several files and folders that comprise the server that is built to serve the task into the M\+Turk UI\+:

\subsubsection*{Folders}


\begin{DoxyItemize}
\item {\bfseries dev/}\+: contains the react frontend components that comprise the frontend, as well as the css and main javascript file that ties it all together. Of these, the {\ttfamily dev/components/core\+\_\+components.\+jsx} file is likely the most interesting, as it contains the frontend components that are rendered in a normal task. To replace them, you\textquotesingle{}ll need to create a {\ttfamily custom.\+jsx} file following the formatting in the dummy version in the same folder. See an example of this in the {\ttfamily react\+\_\+task\+\_\+demo} task in the {\ttfamily parlai/mturk/tasks} folder.
\item {\bfseries server/}\+: contains the package that is actually served by the heroku server for the task. {\ttfamily server.\+js} is what ends up running on the heroku server, which is a simple one-\/page server with a router for socket messages. The {\ttfamily \hyperlink{mturk_2core_2server__utils_8py}{parlai/mturk/core/server\+\_\+utils.\+py}} file handles building the components from dev into a new copy of this folder when launching a new task.
\end{DoxyItemize}

\subsubsection*{Files}

The rest of the files are associated with the process of building the finalized javascript main via node and outputting it into the server directory.


\begin{DoxyItemize}
\item $\ast$$\ast$.babelrc$\ast$$\ast$\+: links presets and plugins required for babel to transpile the react jsx files into pure js.
\item {\bfseries package.\+json}\+: contains the build dependencies and directs the main build process to run the contents of the webpack config file.
\item {\bfseries webpack.\+config.\+js}\+: configures webpack to grab the contents of the {\ttfamily dev} folder and output the final built file to {\ttfamily server/static}. 
\end{DoxyItemize}