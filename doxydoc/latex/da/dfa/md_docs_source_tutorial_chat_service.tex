This is a tutorial for connecting your trained Parl\+AI agents to various chat services in order to allow your models to talk to humans. Humans using chat services (like Facebook Messenger) can be viewed as another type of agent in Parl\+AI and communicate in the standard observation/act dict format.

We currently support the following chat services\+:


\begin{DoxyEnumerate}
\item {\bfseries Browser}
\item {\bfseries Facebook Messenger}
\item {\bfseries Terminal}
\item {\bfseries Web Sockets}
\end{DoxyEnumerate}

You can find more information on how to set up these services below.

Please read here \href{https://github.com/facebookresearch/ParlAI/tree/master/parlai/chat_service}{\tt here} for information on how to set up a new chat service.

\subsection*{Overview}

As stated, humans messaging on chat services can be viewed as a type of agent in Parl\+AI, communicating with models via observations and action dicts. See \href{https://github.com/facebookresearch/ParlAI/blob/master/parlai/chat_service/core/agents.py}{\tt here} for the basic implementation. Human agents are placed in worlds with Parl\+AI agent(s) and possibly other humans, and the world defines how each of these agents interacts.

The chat environment is defined by the {\bfseries task}. A task typically consists of an {\ttfamily Overworld}, which can spawn subtasks (subworlds) or serve as a \char`\"{}main menu\char`\"{}, allowing people to pick from multiple conversation options. The task definition resides in a config file, {\ttfamily config.\+yml}, in which all all available worlds and any additional commandline arguments.

Here is an example config file for a Messenger chatbot\+:


\begin{DoxyCode}
tasks:
  default:
    onboard\_world: MessengerBotChatOnboardWorld
    task\_world: MessengerBotChatTaskWorld
    timeout: 1800
    agents\_required: 1
task\_name: chatbot
world\_module: parlai.chat\_service.tasks.chatbot.worlds
overworld: MessengerOverworld
page\_id: 1 # Configure Your Own Page
max\_workers: 30
opt:  # Additional model opts go here
  debug: True
  model: legacy:seq2seq:0
  model\_file: models:convai2/seq2seq/convai2\_self\_seq2seq\_model
  override:
    model: legacy:seq2seq:0
\end{DoxyCode}


After following the set-\/up instructions (detailed below), this task could be run with the following command from the {\ttfamily parlai/chat\+\_\+service/services/messenger} directory\+: 
\begin{DoxyCode}
python run.py --config-path ../../tasks/chatbot/config.yml
\end{DoxyCode}


\subsection*{Example Tasks}

As an example, the \href{https://github.com/facebookresearch/ParlAI/blob/master/parlai/chat_service/tasks/overworld_demo/}{\tt Overworld Demo} displays three separate tasks connected together by an overworld.


\begin{DoxyItemize}
\item The {\ttfamily echo} task is a simple example of an echo bot, and shows the functionality and flow of a simple single-\/person world.
\item The {\ttfamily onboard data} task is an example that shows how an onboarding world can collect information that is later exposed in the active task world.
\item The {\ttfamily chat} task is an example of a task that requires multiple users, and shows how many people can be connected together in an instance of a world and then returned to the overworld upon completion of a task.
\end{DoxyItemize}

In addition to the overworld demo, the following example tasks are provided\+:


\begin{DoxyItemize}
\item \href{https://github.com/facebookresearch/ParlAI/blob/master/parlai/chat_service/tasks/chatbot/}{\tt Generic Chatbot}\+: Allow conversations with any Parl\+AI models, for instance the \href{https://github.com/facebookresearch/ParlAI/tree/master/projects/personachat}{\tt Persona\+Chat} model.
\item \href{https://github.com/facebookresearch/ParlAI/blob/master/parlai/chat_service/tasks/qa_data_collection/}{\tt QA Data Collection}\+: collect questions and answers from people, given a random Wikipedia paragraph from S\+Qu\+AD.
\end{DoxyItemize}

\subsubsection*{Creating your Own Task}

To create your own task, start with reading the tutorials on the provided examples, and then copy and modify the example {\ttfamily worlds.\+py} and {\ttfamily config.\+yml} files to create your task.

{\bfseries A few things to keep in mind\+:}


\begin{DoxyEnumerate}
\item A conversation ends when a call between {\ttfamily parley} calls to {\ttfamily episode\+\_\+done} returns True.
\item Tasks with an overworld should return the name of the world that they want to queue a user into from the {\ttfamily parley} call in which the user makes that selection to enter a world.
\item Tasks with no overworld will immediately attempt to put a user into the queue for the default task onboarding world or actual task world (if no onboarding world exists), and will do so again following the completion of a world (via {\ttfamily episode\+\_\+done}).
\end{DoxyEnumerate}
\begin{DoxyEnumerate}
\item To collect the conversation, data should be collected during every {\ttfamily parley} and saved during the {\ttfamily world.\+shutdown} call. You must inform the user of the fact that the data is being collected as well as your intended use.
\item Finally, if you wish to use and command line arguments as you would in Parl\+AI, specify those in the {\ttfamily opt} section of the config file.
\end{DoxyEnumerate}

\subsection*{Available Chat Services}

\subsubsection*{Browser}

This allows you to participate in a Parl\+AI world as an agent using a local browser. This extends the {\ttfamily websocket} chat service implementation to run a server locally, which you can send and receive messages using a browser.

\paragraph*{Setup}


\begin{DoxyEnumerate}
\item Run\+: {\ttfamily python \hyperlink{parlai_2chat__service_2services_2browser__chat_2run_8py}{parlai/chat\+\_\+service/services/browser\+\_\+chat/run.\+py} -\/-\/config-\/path path/to/config.\+yml -\/-\/port P\+O\+R\+T\+\_\+\+N\+U\+M\+B\+ER}
\item Run\+: {\ttfamily python client.\+py -\/-\/port P\+O\+R\+T\+\_\+\+N\+U\+M\+B\+ER}
\item Interact
\end{DoxyEnumerate}

{\bfseries Example command\+:} {\ttfamily python \hyperlink{parlai_2chat__service_2services_2browser__chat_2run_8py}{parlai/chat\+\_\+service/services/browser\+\_\+chat/run.\+py} -\/-\/config-\/path parlai/chat\+\_\+service/tasks/chatbot/config.\+yml -\/-\/port 10001}

If no port number is specified in {\ttfamily -\/-\/port} then the default port used will be {\ttfamily 34596}. If specifying, ensure both port numbers match on client and server side.

\subsubsection*{Facebook Messenger}

This allows you to chat with a Parl\+AI model on Facebook Messenger.



\paragraph*{Setup}


\begin{DoxyItemize}
\item Parl\+AI\textquotesingle{}s Messenger functionality requires a free heroku account which can be obtained \href{https://signup.heroku.com/}{\tt here}. Running any Parl\+AI Messenger operation will walk you through linking the two.
\item Running and testing a bot on the \href{https://developers.facebook.com/docs/messenger-platform}{\tt Facebook Messenger Platform} for yourself will require following the guide to set up a \href{https://developers.facebook.com/docs/messenger-platform/getting-started/app-setup}{\tt Facebook App} for Messenger. Skip the set up your webhook step, as Parl\+AI will do it for you.
\item When the guide asks you to configure your webhook U\+RL, you\textquotesingle{}re ready to run the task. This can be done by running the {\ttfamily run.\+py} file in with python.
\item After the heroku server is setup, the script will print out your webhook U\+RL to the console, this should be used to continue the tutorial. The default verify token is {\ttfamily Messenger4\+Parl\+AI}. This U\+RL should be added in the Webhook section. The webhook subscription fields should also be edited to subscribe to the {\ttfamily messages} field.
\item On the first run, the page will ask you for a \char`\"{}\+Page Access Token,\char`\"{} which is also referred to on the messenger setup page. Paste this in to finish the setup. You should now be able to communicate with your Parl\+AI world by messaging your page.
\item To open up your bot for the world to use, you\textquotesingle{}ll need to submit your bot for approval from the \href{https://developers.facebook.com/apps/}{\tt Developer Dashboard}.
\end{DoxyItemize}

{\bfseries Note\+:} When running a new task from a different directory, the webhook url will change. You will need to update this in the developer console from the webhook settings using \char`\"{}edit subscription.\char`\"{} Your Page Access token should not need to be changed unless you want to use a different page.

Additional flags can be used (you can also specify these in the {\ttfamily config.\+yml} file)\+:


\begin{DoxyItemize}
\item {\ttfamily -\/-\/password $<$value$>$} requires that a user sends the message contained in {\ttfamily value} to the bot in order to access the rest of the communications.
\item {\ttfamily -\/-\/force-\/page-\/token} forces the script to request a new page token from you, allowing you to switch what page you\textquotesingle{}re running your bot on.
\item {\ttfamily -\/-\/verbose} and {\ttfamily -\/-\/debug} should be used before reporting problems that arise that appear unrelated to your world, as they expose more of the internal state of the messenger manager.
\end{DoxyItemize}

{\bfseries Other things to keep in mind when creating your Messenger tasks\+:}
\begin{DoxyItemize}
\item Your world can utilize the complete set of \href{https://developers.facebook.com/docs/messenger-platform/send-messages/templates}{\tt Facebook Messenger Templates} by putting the formatted data in the \textquotesingle{}payload\textquotesingle{} field of the observed action.
\item Quick replies can be attached to any action, the {\ttfamily Messenger\+Overworld} of the \href{https://github.com/facebookresearch/ParlAI/blob/master/parlai/chat_service/tasks/overworld_demo/}{\tt Overworld Demo} displays this functionality.
\end{DoxyItemize}

\subsubsection*{Terminal}

This allows you to participate in a Parl\+AI world as an agent using the terminal. This extends the {\ttfamily websocket} chat service implementation to run a server locally, which you can send and receive messages from using the terminal.

\paragraph*{Setup}


\begin{DoxyEnumerate}
\item Run\+: {\ttfamily python \hyperlink{parlai_2chat__service_2services_2terminal__chat_2run_8py}{parlai/chat\+\_\+service/services/terminal\+\_\+chat/run.\+py} -\/-\/config-\/path path/to/config.\+yml -\/-\/port P\+O\+R\+T\+\_\+\+N\+U\+M\+B\+ER}
\item Run\+: {\ttfamily python client.\+py -\/-\/port P\+O\+R\+T\+\_\+\+N\+U\+M\+B\+ER}
\item Interact
\end{DoxyEnumerate}

{\bfseries Example command\+:} {\ttfamily python \hyperlink{parlai_2chat__service_2services_2terminal__chat_2run_8py}{parlai/chat\+\_\+service/services/terminal\+\_\+chat/run.\+py} -\/-\/config-\/path parlai/chat\+\_\+service/tasks/chatbot/config.\+yml -\/-\/port 10001}

If no port number is specified in {\ttfamily -\/-\/port} then the default port used will be {\ttfamily 34596}. If specifying, ensure both port numbers match on client and server side.

\subsubsection*{Web Sockets}

See {\bfseries Browser} above for an example implementation of a websockets-\/based chat service. You can view the code \href{https://github.com/facebookresearch/ParlAI/tree/master/parlai/chat_service/services/browser_chat}{\tt here}.

\subsubsection*{Adding a New Chat Service}

For full instructions on adding a new chat service to Parl\+AI, please read https\+://github.com/facebookresearch/\+Parl\+A\+I/tree/master/parlai/chat\+\_\+service/\+R\+E\+A\+D\+M\+E.\+md \char`\"{}here\char`\"{}. 