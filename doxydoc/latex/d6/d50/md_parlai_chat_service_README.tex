This is an instruction manual to be used as reference on how to configure Parl\+AI for an arbitrary chat service.

\subsection*{Message Format}

To maintain consistency we are trying to enforce a deterministic message format throughout this task. If a particular chat service doesn\textquotesingle{}t adhere to this format, one must use {\ttfamily restructure\+\_\+message()} to adapt the messages to this format before the messages are used. The message format is defined below\+: 
\begin{DoxyCode}
\{
  mid: # ID of this message
  recipient: \{
    id: #id of message recipient
  \}
  sender: \{
    id: #id of message sender
  \}
  text: # text of the message
  attachment: # attachment of the message
\}
\end{DoxyCode}
 \subsubsection*{Additional Message fields}

These define a non-\/exhaustive list of keys that one could use in the message dict for ease-\/of-\/use
\begin{DoxyItemize}
\item messaging\+\_\+type\+: whether the message is a text message or an image upload \mbox{[}R\+E\+S\+P\+O\+N\+SE, U\+P\+D\+A\+TE\mbox{]}
\item quickreplies\+: Auto-\/suggested replies
\item persona\+\_\+id\+: id of the persona that is interacting
\item name\+: Display name of the user
\item profile\+\_\+picture\+\_\+url\+: U\+RL to the profile picture of the user
\end{DoxyItemize}

\subsection*{Config}

Below is the standard config format and hierarchy to be followed across all chat services\+:

```config
\begin{DoxyItemize}
\item tasks\+: \# List of available tasks/worlds for a user to enter
\begin{DoxyItemize}
\item $<$task 1 name$>$
\begin{DoxyItemize}
\item onboard\+\_\+world\+: \# World in which user is first send upon selecting the task. Can collect necessary data from user in this world, as well as provide instructions etc.
\item task\+\_\+world\+: \# Actual task world
\item timeout\+: \# Agent message timeout -\/ how long to wait for the agent to send a message before assuming they have disconnected.
\item agents\+\_\+required\+: \# Number of agents required to run the task world, e.\+g. 2 for a two player game, 1 for a one-\/player experience, etc.
\end{DoxyItemize}
\item $<$task 2 name$>$
\begin{DoxyItemize}
\item onboard\+\_\+world\+:
\item task\+\_\+world\+:
\item timeout\+:
\item agents\+\_\+required\+:
\end{DoxyItemize}
\end{DoxyItemize}
\item task\+\_\+name\+: \# name of the overall task
\item world\+\_\+module\+: \# module in which all of the worlds exist (relative path i.\+e. {\ttfamily \hyperlink{namespaceparlai_1_1chat__service}{parlai.\+chat\+\_\+service}....}
\item overworld\+: \# Name of the overworld; where the agent is first sent upon messaging the service
\item max\+\_\+workers\+: \# Maximum number of workers that can be in task worlds at any given moment
\item opt\+: \# Additional model opts go here. Below are example opts that one could normally pass to parlai
\begin{DoxyItemize}
\item password\+: \# Password for messaging service, if this is wanted
\item debug\+: \# whether to set debug mode
\item model\+: \# Name of model, if you want to load a model
\item model\+\_\+file\+: \# path to model file, if you want to load a model
\item override\+:
\begin{DoxyItemize}
\item model\+: \# overrides for model
\end{DoxyItemize}
\end{DoxyItemize}
\item additional\+\_\+args\+: \# Additional chat service specific args go here
\begin{DoxyItemize}
\item service\+\_\+reference\+\_\+id\+: 1 \# Facebook Page id (if Messenger, else don\textquotesingle{}t include this field)
\item {\itshape any other args needed by $<$chat\+\_\+service$>$} ```
\end{DoxyItemize}
\end{DoxyItemize}

As one can notice, most of the format is the same as how it already exists for messenger with the exception of having {\ttfamily additional\+\_\+args\+:} as a field in our config. This has been introduced to provide flexibility of parsing any additional arguments a chat service may need whilst preserving the previously existing necessary args. Note however that {\ttfamily page\+\_\+id} has been shifted to this section to maintain coherence.

\subsection*{Manager}

To implement your own service, you will need to subclass the {\ttfamily Chat\+Service\+Manager} in {\ttfamily \hyperlink{chat__service__manager_8py}{chat\+\_\+service/core/chat\+\_\+service\+\_\+manager.\+py}}. The {\ttfamily Chat\+Service\+Manager} handles a lot of the abstraction for you, so you\textquotesingle{}ll only need to subclass a few of the methods in there when implementing your own service.

Some notable essentials below (note not all abstract methods are specified)\+:


\begin{DoxyEnumerate}
\item {\ttfamily Chat\+Service\+Message\+Sender} -\/ The message sender is useful for wrapping requests with additional functions. E.\+g., in the Messenger service, the {\ttfamily Message\+Sender} implements {\ttfamily send\+\_\+read} and {\ttfamily typing\+\_\+on} functions which are called upon message sends.
\item {\ttfamily parse\+\_\+additional\+\_\+args} -\/ This is the function for parsing the additional args from the config specified above
\item {\ttfamily \+\_\+complete\+\_\+setup} -\/ Complete any additional setup items; should be called in initialization.
\item {\ttfamily \+\_\+load\+\_\+model} -\/ This function should load the model, if necessary. This varies per chat service.
\item {\ttfamily restructure\+\_\+message} -\/ If messages in your service are not the same format as specified in {\itshape Message Format} above, please use this method to restructure the message.
\item {\ttfamily setup\+\_\+server} -\/ sets up the chat service server.
\item {\ttfamily setup\+\_\+socket} -\/ sets up the socket to start communicating with users.
\item {\ttfamily observe\+\_\+message} -\/ Send a message thru the message manager to a corresponding user/agent.
\end{DoxyEnumerate}

\subsection*{Manager Utils}

\subsubsection*{Socket}

The {\ttfamily Chat\+Service\+Message\+Socket} is a wrapper around websockets to forward messages from a remote server to a {\ttfamily Chat\+Service\+Manager}.

\subsubsection*{Runner}

The {\ttfamily Chat\+Service\+World\+Runner} is the actual class for handling running worlds, overworlds, etc.

\subsection*{Agents}

Finally, once you have implemented your {\ttfamily Chat\+Service\+Manager}, you will need to implement a {\ttfamily Chat\+Service\+Agent}, which is specific for each chat service. The following are notable functions to implement in your own {\ttfamily Chat\+Service\+Agent}.


\begin{DoxyEnumerate}
\item {\ttfamily observe} -\/ Implement this function to receive messages from the manager.
\item {\ttfamily put\+\_\+data} -\/ Puts data into the Agent\textquotesingle{}s message queue if it hasn\textquotesingle{}t already been seen. 
\end{DoxyEnumerate}