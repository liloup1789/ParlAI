\subsection*{Fonctionnement minimal \+:}

\subsubsection*{Créer une tâche -\/$>$ build.\+py pour télécharger le corpus et l\textquotesingle{}installer dans le répertoire /data. Si déjà présent, ne rien faire. -\/$>$ agents.\+py qui crée les teachers \+: ce sont eux qui parsent les données (en mode entraînement, ils stockent le text + les labels) et qui se mettent automatiquement en mode question-\/answering, ou dialogue. Ce sont eux qui vont jouer le rôle de l\textquotesingle{}utilisateur pour entraîner les agents. Parl\+AI fournit tout un tas de teachers, le nôtre est une classe fille de Parl\+A\+I\+D\+Ialog\+Teacher, je te laisse regarder la doc.}

\subsubsection*{Utiliser Parl\+AI -\/$>$ dans le répertoire /examples il y\textquotesingle{}a tout un tas de scripts (relativement bourrins) qui servent à comprendre comment marche le système de monde. \hyperlink{namespacedisplay__data}{display\+\_\+data} est un bon exemple \+:}


\begin{DoxyItemize}
\item $\ast$$\ast$ On crée un monde qui, par défaut, s\textquotesingle{}accorde avec nos données (class Dialogue\+World (World)), cela implémente la méthode world.\+parlai() -\/$>$ agent1 ( le teacher).act() / agent2 (l\textquotesingle{}apprenant).observe() / agent2.\+act / agent1.\+observe.
\item $\ast$$\ast$ les méthodes act et observe sont propres à chaque agent. Dans le cas du teacher, nous utilisons celles par défaut. Là où c\textquotesingle{}est intéressant c\textquotesingle{}est pour l\textquotesingle{}apprenant.
\item $\ast$$\ast$ Pour charger un agent apprenant, Parl\+AI implémente un certain nombre de modèles d\textquotesingle{}agents. Nous utilisons pour \hyperlink{namespacedisplay__data}{display\+\_\+data} l\textquotesingle{}agent Reapeat\+Label Agent, qui lit ce que lui donne le teacher et le recrache tel un perroquet des îles. \subsubsection*{Dans le script train\+\_\+data.\+py nous utilisons l\textquotesingle{}exemple Mem\+N\+N\+Agent (aka memory\+Network Agent) qui va se comporter d\textquotesingle{}une façon particulière par rapport à d\textquotesingle{}autres agents d\textquotesingle{}apprentissage. Les résultats de l\textquotesingle{}apprentissage sont sauvegardés dans un fichier de modèle.}
\end{DoxyItemize}


\begin{DoxyItemize}
\item $\ast$$\ast$ dans le main, une boucle while not \char`\"{}\+E\+P\+O\+C\+H D\+O\+N\+E\char`\"{} enveloppe world.\+parlai. Tant que tous les dialogues ne sont pas appris, cela continue.
\item $\ast$$\ast$ Le framework derrière tous les agents d\textquotesingle{}apprentissage est Py\+Torch, il y\textquotesingle{}a d\textquotesingle{}ailleurs une classe Torch\+Agent dont la plupart des agents héritent. Dont le Mem\+NN Agent.
\end{DoxyItemize}

\subsubsection*{Tu peux trouver les scripts que j\textquotesingle{}ai faits entièrement from scratch dans /parlai/tasks/cologne. les données sont dans /data/cologne}



\href{https://github.com/facebookresearch/ParlAI/blob/master/LICENSE}{\tt } \href{https://circleci.com/gh/facebookresearch/ParlAI/tree/master}{\tt } https\+://github.com/facebookresearch/\+Parl\+A\+I/blob/master/\+C\+O\+N\+T\+R\+I\+B\+U\+T\+I\+N\+G.\+md \char`\"{}!\mbox{[}\+P\+Rs Welcome\mbox{]}(https\+://img.\+shields.\+io/badge/\+P\+Rs-\/welcome-\/brightgreen.\+svg)\char`\"{} \href{https://twitter.com/parlai_parley}{\tt } 



\href{http://parl.ai}{\tt Parl\+AI} (pronounced “par-\/lay”) is a python framework for sharing, training and testing dialogue models, from open-\/domain chitchat to V\+QA (Visual Question Answering).

Its goal is to provide researchers\+:


\begin{DoxyItemize}
\item {\bfseries 80+ popular datasets available all in one place, with the same A\+PI}, among them \href{https://arxiv.org/abs/1801.07243}{\tt Persona\+Chat}, \href{https://arxiv.org/abs/1710.03957}{\tt Daily\+Dialog}, \href{https://openreview.net/forum?id=r1l73iRqKm}{\tt Wizard of Wikipedia}, \href{https://arxiv.org/abs/1811.00207}{\tt Empathetic Dialogues}, \href{https://rajpurkar.github.io/SQuAD-explorer/}{\tt S\+Qu\+AD}, \href{http://www.msmarco.org/}{\tt MS M\+A\+R\+CO}, \href{https://www.aclweb.org/anthology/D18-1241}{\tt Qu\+AC}, \href{https://hotpotqa.github.io/}{\tt Hotpot\+QA}, \href{https://arxiv.org/abs/1506.03340}{\tt Q\+A\+C\+NN \& Q\+A\+Daily\+Mail}, \href{https://arxiv.org/abs/1511.02301}{\tt C\+BT}, \href{https://arxiv.org/abs/1610.00956}{\tt Book\+Test}, \href{https://arxiv.org/abs/1605.07683}{\tt b\+AbI Dialogue tasks}, \href{https://arxiv.org/abs/1506.08909}{\tt Ubuntu Dialogue}, \href{http://opus.lingfil.uu.se/OpenSubtitles.php}{\tt Open\+Subtitles}, \href{https://arxiv.org/abs/1811.00945}{\tt Image Chat}, \href{http://visualqa.org/}{\tt V\+QA}, \href{https://arxiv.org/abs/1611.08669}{\tt Vis\+Dial} and \href{http://cs.stanford.edu/people/jcjohns/clevr/}{\tt C\+L\+E\+VR}. See the complete list \href{https://github.com/facebookresearch/ParlAI/blob/master/parlai/tasks/task_list.py}{\tt here}.
\item a wide set of \href{https://www.parl.ai/docs/agents_list.html}{\tt {\bfseries reference models}} -- from retrieval baselines to transformers.
\item a large \href{https://parl.ai/docs/zoo.html}{\tt zoo of {\bfseries pretrained models}} ready to use off-\/the-\/shelf
\item seamless {\bfseries integration of \href{https://www.mturk.com/mturk/welcome}{\tt Amazon Mechanical Turk}} for data collection and human evaluation
\item {\bfseries integration with \href{http://www.parl.ai/docs/tutorial_messenger.html}{\tt Facebook Messenger}} to connect agents with humans in a chat interface
\item a large range of {\bfseries helpers to create your own agents} and train on several tasks with {\bfseries multitasking}
\item {\bfseries multimodality}, some tasks use text and images
\end{DoxyItemize}

Parl\+AI is described in the following paper\+: \href{https://arxiv.org/abs/1705.06476}{\tt “\+Parl\+A\+I\+: A Dialog Research Software Platform\char`\"{}, ar\+Xiv\+:1705.\+06476$<$/a$>$
or see these $<$a href=\char`\"{}https\+://drive.\+google.\+com/file/d/1\+Jf\+U\+W4\+A\+Vrj\+Sp8\+X8\+Fp0\+\_\+r\+T\+T\+Ro\+Lx\+Uf\+W0a\+Um/view?usp=sharing"$>$more up-\/to-\/date slides}.

See the https\+://github.com/facebookresearch/\+Parl\+A\+I/blob/master/\+N\+E\+W\+S.\+md \char`\"{}news page\char`\"{} for the latest additions \& updates, and the website \href{http://parl.ai}{\tt http\+://parl.\+ai} for further docs.



\subsection*{Installing Parl\+AI}

Parl\+AI currently requires Python3 and \href{https://pytorch.org}{\tt Pytorch} 1.\+1 or newer. Dependencies of the core modules are listed in {\ttfamily requirement.\+txt}. Some models included (in {\ttfamily parlai/agents}) have additional requirements.

Run the following commands to clone the repository and install Parl\+AI\+:


\begin{DoxyCode}
git clone https://github.com/facebookresearch/ParlAI.git ~/ParlAI
cd ~/ParlAI; python setup.py develop
\end{DoxyCode}


This will link the cloned directory to your site-\/packages.

This is the recommended installation procedure, as it provides ready access to the examples and allows you to modify anything you might need. This is especially useful if you if you want to submit another task to the repository.

All needed data will be downloaded to {\ttfamily $\sim$/\+Parl\+A\+I/data}, and any non-\/data files if requested will be downloaded to {\ttfamily $\sim$/\+Parl\+A\+I/downloads}. If you need to clear out the space used by these files, you can safely delete these directories and any files needed will be downloaded again.

\subsection*{Documentation}


\begin{DoxyItemize}
\item \href{https://parl.ai/docs/tutorial_quick.html}{\tt Quick Start}
\item \href{https://parl.ai/docs/tutorial_basic.html}{\tt Basics\+: world, agents, teachers, action and observations}
\item \href{https://parl.ai/docs/tasks.html}{\tt List of available tasks/datasets}
\item \href{http://www.parl.ai/docs/tutorial_task.html}{\tt Creating a dataset/task}
\item \href{./parlai/agents}{\tt List of available agents}
\item \href{https://parl.ai/docs/tutorial_seq2seq.html#}{\tt Creating a new agent}
\item \href{https://parl.ai/docs/zoo.html}{\tt Model zoo (pretrained models)}
\item \href{http://parl.ai/docs/tutorial_mturk.html}{\tt Plug into M\+Turk}
\item \href{http://parl.ai/docs/tutorial_messenger.html}{\tt Plug into Facebook Messenger}
\end{DoxyItemize}

\subsection*{Examples}

A large set of examples can be found in \href{./examples}{\tt this directory}. Here are a few of them. Note\+: If any of these examples fail, check the \href{#requirements}{\tt requirements section} to see if you have missed something.

Display 10 random examples from the S\+Qu\+AD task 
\begin{DoxyCode}
python examples/display\_data.py -t squad
\end{DoxyCode}


Evaluate an IR baseline model on the validation set of the Personachat task\+: 
\begin{DoxyCode}
python examples/eval\_model.py -m ir\_baseline -t personachat -dt valid
\end{DoxyCode}


Train a single layer transformer on Persona\+Chat (requires pytorch and torchtext). Detail\+: embedding size 300, 4 attention heads, 2 epochs using batchsize 64, word vectors are initialized with fasttext and the other elements of the batch are used as negative during training. 
\begin{DoxyCode}
python examples/train\_model.py -t personachat -m transformer/ranker -mf /tmp/model\_tr6 --n-layers 1
       --embedding-size 300 --ffn-size 600 --n-heads 4 --num-epochs 2 -veps 0.25 -bs 64 -lr 0.001 --dropout 0.1
       --embedding-type fasttext\_cc --candidates batch
\end{DoxyCode}


\subsection*{Code Organization}

The code is set up into several main directories\+:


\begin{DoxyItemize}
\item \href{./parlai/core}{\tt {\bfseries core}}\+: contains the primary code for the framework
\item \href{./parlai/agents}{\tt {\bfseries agents}}\+: contains agents which can interact with the different tasks (e.\+g. machine learning models)
\item \href{./parlai/examples}{\tt {\bfseries examples}}\+: contains a few basic examples of different loops (building dictionary, train/eval, displaying data)
\item \href{./parlai/tasks}{\tt {\bfseries tasks}}\+: contains code for the different tasks available from within Parl\+AI
\item \href{./parlai/mturk}{\tt {\bfseries mturk}}\+: contains code for setting up Mechanical Turk, as well as sample M\+Turk tasks
\item \href{./parlai/chat_service/services/messenger}{\tt {\bfseries messenger}}\+: contains code for interfacing with Facebook Messenger
\item \href{./parlai/zoo}{\tt {\bfseries zoo}}\+: contains code to directly download and use pretrained models from our model zoo
\end{DoxyItemize}

\subsection*{Support}

If you have any questions, bug reports or feature requests, please don\textquotesingle{}t hesitate to post on our \href{https://github.com/facebookresearch/ParlAI/issues}{\tt Github Issues page}.

\subsection*{The Team}

Parl\+AI is currently maintained by Emily Dinan, Dexter Ju, Margaret Li, Spencer Poff, Pratik Ringshia, Stephen Roller, Kurt Shuster, Eric Michael Smith, Jack Urbanek, Jason Weston, and Mary Williamson.

Former major contributors and maintainers include Alexander H. Miller, Will Feng, Adam Fisch, Jiasen Lu, Antoine Bordes, Devi Parikh, Dhruv Batra, Filipe de Avila Belbute Peres and Chao Pan.

\subsection*{Citation}

Please cite the \href{https://arxiv.org/abs/1705.06476}{\tt ar\+Xiv paper} if you use Parl\+AI in your work\+:


\begin{DoxyCode}
@article\{miller2017parlai,
  title=\{ParlAI: A Dialog Research Software Platform\},
  author=\{\{Miller\}, A.~H. and \{Feng\}, W. and \{Fisch\}, A. and \{Lu\}, J. and \{Batra\}, D. and \{Bordes\}, A. and
       \{Parikh\}, D. and \{Weston\}, J.\},
  journal=\{arXiv preprint arXiv:\{1705.06476\}\},
  year=\{2017\}
\}
\end{DoxyCode}


\subsection*{License}

Parl\+AI is M\+IT licensed. See the L\+I\+C\+E\+N\+SE file for details. 